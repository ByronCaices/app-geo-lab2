\section{Conclusiones}

El presente estudio cuantificó exitosamente 1,689 ha de urbanización en Peñaflor (2018-2024) mediante teledetección Sentinel-2, detectando una tasa anual de 1.16\% (282 ha/año) que supera en 226\% las proyecciones del Plan Regulador Comunal. Se identificó pérdida de 2,086 ha de vegetación concentrada en 18 zonas críticas que representan el 58\% de la urbanización total. La concordancia entre los 3 métodos de detección implementados (diferencias $<$10\%) y validación con Google Earth (90\% precisión, superando estándares USGS) validan la robustez del análisis. La correlación NDVI-NDBI de -0.87 (p$<$0.001) confirma que la urbanización ocurre mediante conversión directa de áreas vegetadas.

Se cumplieron exitosamente los 6 objetivos: adquisición de 4 imágenes Sentinel-2 con nubosidad $<$5\%, cálculo de 16 capas de índices espectrales (161 MB), implementación de 3 métodos de detección con precisión $>$85\%, análisis zonal sobre grilla de 100 celdas, desarrollo de dashboard Streamlit funcional, y validación empírica completada. Las contribuciones incluyen: (1) pipeline Python reproducible open source replicable en 20+ comunas RM, (2) diseño de grilla sistemática que mitiga el Problema de Unidad de Área Modificable (MAUP), (3) Índice de Transformación compuesto integrando urbanización, pérdida vegetativa y cambio NDBI, y (4) dashboard web que democratiza acceso a datos de teledetección. Aplicaciones directas incluyen evidencia empírica para actualización del PRC (1.2\%/año), identificación de 18 zonas prioritarias para intervención, línea base cuantitativa para monitoreo futuro, y metodología replicable en comunas periurbanas.

El análisis presenta limitaciones técnicas (resolución 10 m insuficiente para urbanización dispersa $<$100 m², confusión espectral 5\%, serie temporal corta de 6 años), metodológicas (umbrales fijos, grilla artificial, variabilidad estacional) y de datos (ausencia de capas auxiliares oficiales, acceso restringido a estadísticas municipales, nubosidad residual ~1\%). Se recomiendan acciones a corto plazo (actualizar PRC, adopción de dashboard, Evaluación Ambiental Estratégica en zonas críticas), mediano plazo (integrar LiDAR, modelar proyecciones 2030-2040, cuantificar impacto en servicios ecosistémicos) y largo plazo (pipeline automatizado con alertas cada 10 días, integración con sistema catastral, red de sensores IoT).

El proyecto demuestra que teledetección satelital, análisis espacial automatizado y visualización interactiva constituyen herramientas poderosas y accesibles para monitoreo territorial en contextos de periurbanización acelerada. La capacidad de cuantificar transformaciones con 90\% de precisión usando datos gratuitos (Sentinel-2, Google Earth Engine) y software open source (Python, Streamlit, GeoPandas) democratiza el acceso a información geoespacial crítica para planificación territorial basada en evidencia. La metodología desarrollada es escalable y replicable, habilitando su aplicación en las 52 comunas de la Región Metropolitana para generación de un atlas metropolitano de cambio de uso de suelo que fortalezca la gobernanza territorial regional.
