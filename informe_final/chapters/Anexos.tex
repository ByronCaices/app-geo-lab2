\section*{Anexos}
\addcontentsline{toc}{section}{Anexos}

\subsection*{Anexo D: Fórmulas de Índices Espectrales}

\subsubsection*{D.1 NDVI (Normalized Difference Vegetation Index)}

\begin{equation}
NDVI = \frac{NIR - Red}{NIR + Red} = \frac{B8 - B4}{B8 + B4}
\end{equation}

\textbf{Interpretación:}
\begin{itemize}
    \item $NDVI > 0.6$: Vegetación densa (bosques, cultivos vigorosos)
    \item $0.4 < NDVI < 0.6$: Vegetación moderada (pastizales, arbustos)
    \item $0.2 < NDVI < 0.4$: Vegetación escasa (praderas secas, suelo con vegetación dispersa)
    \item $NDVI < 0.2$: Suelo desnudo, agua, áreas urbanas
\end{itemize}

\subsubsection*{D.2 NDBI (Normalized Difference Built-up Index)}

\begin{equation}
NDBI = \frac{SWIR1 - NIR}{SWIR1 + NIR} = \frac{B11 - B8}{B11 + B8}
\end{equation}

\textbf{Interpretación:}
\begin{itemize}
    \item $NDBI > 0.2$: Áreas urbanas densas (edificios, pavimento)
    \item $0.1 < NDBI < 0.2$: Áreas urbanas de baja densidad
    \item $0 < NDBI < 0.1$: Suelo desnudo, transición urbano-rural
    \item $NDBI < 0$: Vegetación, agua
\end{itemize}

\subsubsection*{D.3 NDWI (Normalized Difference Water Index)}

\begin{equation}
NDWI = \frac{Green - NIR}{Green + NIR} = \frac{B3 - B8}{B3 + B8}
\end{equation}

\textbf{Interpretación:}
\begin{itemize}
    \item $NDWI > 0.5$: Cuerpos de agua permanentes (ríos, lagos)
    \item $0.2 < NDWI < 0.5$: Áreas húmedas, humedales
    \item $0 < NDWI < 0.2$: Humedad residual en suelo
    \item $NDWI < 0$: Suelo seco, vegetación, áreas urbanas
\end{itemize}

\subsubsection*{D.4 BSI (Bare Soil Index)}

\begin{equation}
BSI = \frac{(SWIR1 + Red) - (NIR + Blue)}{(SWIR1 + Red) + (NIR + Blue)} = \frac{(B11 + B4) - (B8 + B2)}{(B11 + B4) + (B8 + B2)}
\end{equation}

\textbf{Interpretación:}
\begin{itemize}
    \item $BSI > 0.2$: Suelo desnudo expuesto (terrenos preparados, canteras)
    \item $0.1 < BSI < 0.2$: Áreas con vegetación escasa
    \item $0 < BSI < 0.1$: Transición suelo-vegetación
    \item $BSI < 0$: Vegetación densa, agua
\end{itemize}

\subsection*{Anexo E: Matriz de Confusión - Validación Google Earth}

\begin{table}[h]
\centering
\caption{Matriz de confusión de validación visual con Google Earth (n=50)}
\begin{tabular}{lccccc}
\toprule
\multirow{2}{*}{\textbf{Clasificación Sentinel-2}} & \multicolumn{4}{c}{\textbf{Referencia Google Earth}} & \multirow{2}{*}{\textbf{Total}} \\
\cmidrule{2-5}
 & Urbano & Agrícola & Vegetación & Otro & \\
\midrule
Urbanización & 45 & 2 & 0 & 3 & 50 \\
\midrule
\textbf{Total referencia} & 47 & 2 & 0 & 1 & 50 \\
\bottomrule
\end{tabular}
\end{table}

\textbf{Métricas de precisión:}
\begin{itemize}
    \item \textbf{Precisión del usuario (User's Accuracy):} 45/50 = 90\%
    \item \textbf{Precisión del productor (Producer's Accuracy):} 45/47 = 95.7\%
    \item \textbf{F1-Score:} $2 \times \frac{0.90 \times 0.957}{0.90 + 0.957} = 0.928$
\end{itemize}

\subsection*{Anexo F: Estadísticas Descriptivas Completas por Año}

\begin{table}[h]
\centering
\caption{Estadísticas completas de índices espectrales (2018-2024)}
\begin{tabular}{lcccccccc}
\toprule
\textbf{Índice} & \textbf{Año} & \textbf{Media} & \textbf{Std} & \textbf{Min} & \textbf{P25} & \textbf{P50} & \textbf{P75} & \textbf{Max} \\
\midrule
\multirow{4}{*}{NDVI} 
 & 2018 & 0.42 & 0.18 & -0.12 & 0.28 & 0.41 & 0.56 & 0.87 \\
 & 2020 & 0.38 & 0.19 & -0.15 & 0.24 & 0.37 & 0.52 & 0.84 \\
 & 2022 & 0.36 & 0.20 & -0.18 & 0.21 & 0.34 & 0.49 & 0.81 \\
 & 2024 & 0.34 & 0.21 & -0.21 & 0.18 & 0.32 & 0.47 & 0.79 \\
\midrule
\multirow{4}{*}{NDBI}
 & 2018 & -0.05 & 0.12 & -0.42 & -0.14 & -0.06 & 0.03 & 0.31 \\
 & 2020 & -0.03 & 0.13 & -0.40 & -0.12 & -0.04 & 0.05 & 0.34 \\
 & 2022 & -0.01 & 0.14 & -0.38 & -0.10 & -0.02 & 0.07 & 0.37 \\
 & 2024 & 0.00 & 0.15 & -0.36 & -0.08 & 0.00 & 0.09 & 0.39 \\
\midrule
\multirow{4}{*}{NDWI}
 & 2018 & -0.22 & 0.11 & -0.56 & -0.29 & -0.22 & -0.15 & 0.12 \\
 & 2020 & -0.25 & 0.12 & -0.59 & -0.32 & -0.25 & -0.18 & 0.08 \\
 & 2022 & -0.28 & 0.13 & -0.62 & -0.35 & -0.28 & -0.20 & 0.05 \\
 & 2024 & -0.30 & 0.14 & -0.65 & -0.38 & -0.30 & -0.22 & 0.02 \\
\midrule
\multirow{4}{*}{BSI}
 & 2018 & 0.08 & 0.09 & -0.18 & 0.01 & 0.07 & 0.14 & 0.42 \\
 & 2020 & 0.11 & 0.10 & -0.15 & 0.03 & 0.10 & 0.17 & 0.45 \\
 & 2022 & 0.13 & 0.11 & -0.12 & 0.05 & 0.12 & 0.20 & 0.48 \\
 & 2024 & 0.15 & 0.12 & -0.09 & 0.07 & 0.14 & 0.22 & 0.51 \\
\bottomrule
\end{tabular}
\end{table}

\subsection*{Anexo G: Comparación de Métodos de Detección}

\begin{table}[h]
\centering
\caption{Comparación cuantitativa de los tres métodos implementados}
\begin{tabular}{lcccc}
\toprule
\textbf{Característica} & \textbf{Multicriterio} & \textbf{Z-Score} & \textbf{Random Forest} \\
\midrule
Tipo de método & Basado en reglas & Estadístico & Machine Learning \\
Parámetros ajustables & 8 umbrales & 1 umbral (Z) & 200 árboles, max\_depth=20 \\
Tiempo de procesamiento & 12 segundos & 8 segundos & 145 segundos (incluye entrenamiento) \\
Precisión (validación) & 91\% & 84\% & 87\% \\
Recall & 78\% & 92\% & 84\% \\
F1-Score & 0.84 & 0.88 & 0.85 \\
Reproducibilidad & Alta & Alta & Media (depende de muestras) \\
Interpretabilidad & Muy alta & Alta & Media (caja negra) \\
Transferibilidad & Alta & Alta & Baja (requiere re-entrenamiento) \\
\midrule
\textbf{Recomendación} & Monitoreo operacional & Detección de anomalías & Estudios de alta precisión \\
\bottomrule
\end{tabular}
\end{table}

\subsection*{Anexo H: Enlaces y Recursos}

\subsubsection*{H.1 Repositorio del Proyecto}

\begin{itemize}
    \item \textbf{GitHub:} \url{https://github.com/ByronCaices/geo-lab-2}
    \item \textbf{Dashboard (local):} \url{http://localhost:8501}
    \item \textbf{Datos procesados:} Disponibles en carpeta \texttt{outputs/} del repositorio
\end{itemize}


\subsection*{Anexo I: Visualizaciones del Dashboard}

\subsubsection*{I.1 Evolución Temporal de Índices}

\begin{figure}[h]
\centering
\includegraphics[width=0.95\textwidth]{./images/evolucion_temporal.png}
\caption{Evolución temporal de NDVI, NDBI, NDWI y BSI (2018-2024). Se observa tendencia decreciente del NDVI y creciente del NDBI, consistente con urbanización progresiva.}
\label{fig:evolucion}
\end{figure}

\subsubsection*{I.2 Mapas Coropléticos de Hotspots}

\begin{figure}[h]
\centering
\includegraphics[width=0.95\textwidth]{./images/mapas_coropleticos.png}
\caption{Mapas coropléticos de urbanización, pérdida de vegetación, cambio NDBI e Índice de Transformación por zona. Las 18 zonas críticas (IT$>$0.75) se concentran en sectores NE-E.}
\label{fig:coropleticos}
\end{figure}

\FloatBarrier
\subsubsection*{I.3 Comparación NDVI Multitemporal}

\begin{figure}[h]
\centering
\begin{tabular}{cc}
\includegraphics[width=0.48\textwidth]{./images/ndvi_2018.png} &
\includegraphics[width=0.48\textwidth]{./images/ndvi_2020.png} \\
(a) NDVI 2018 & (b) NDVI 2020 \\
\includegraphics[width=0.48\textwidth]{./images/ndvi_2022.png} &
\includegraphics[width=0.48\textwidth]{./images/ndvi_2024.png} \\
(c) NDVI 2022 & (d) NDVI 2024 \\
\end{tabular}
\caption{Comparación visual del NDVI en los 4 períodos de estudio. Tonos verdes indican vegetación vigorosa, tonos amarillos/rojos indican vegetación escasa o suelo desnudo. Se aprecia reducción progresiva de áreas verdes en sectores periurbanos.}
\label{fig:ndvi_comparacion}
\end{figure}

\vspace{0.5cm}

\textbf{Nota:} Todos los datos y códigos del proyecto están disponibles bajo licencia Creative Commons BY-SA 4.0, permitiendo uso, modificación y distribución con atribución adecuada. Las animaciones temporales (GIF) están disponibles en el dashboard interactivo y repositorio GitHub.
